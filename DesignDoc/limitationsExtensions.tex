%!TEX root = ./main.tex
%----------------------------------------------------------------------------------------
% Limitations and Extensions
%----------------------------------------------------------------------------------------

\section{Limitations and Extensions} % Major section


%------------------------------------------------

\subsection{Limitations} % Sub-section
CAPA does suffer from a number of limitations, some of these are due to it being a static analysis
tool, others being due to restrictions of resources. The major limitations are summarised here.

\subsubsection{CAPA Over Zealously Analyses Code}
This limitation is a combination of the two major limiting factors of the project. Due to the fact
that CAPA is a static analysis tool, much of the information relating to how a program executes is
not available at compile time, and as such many assumptions had to be made in order to provide
thorough reports for the end user. As a result of this a decision was made early on that false
positives were preferable to false negatives, this gave rise to the over zealous nature of CAPA
which identifies regions of a codebase as potentially parallelizable, even though they contain data
dependencies. A solution to this problem would have been to use escape analysis, however the time
investment in developing garbage collecting techniques to perform on the Clang AST was deemed to be
prohibitive, and as such it was not further explored as an option.

\subsubsection{CAPA Has Limited Capture Cases}
CAPA only actively catches four problem classes, the three parallel primitives and matrix
multiplication. Initial goals for CAPA included catching Graph traversals as well as a variety of
BLAS linear algebra operations. Unfortunately early development of CAPA relied heavily on directly
interfacing with the clang AST Matcher interface, rather than through a defined combinator library.
Attempts were made early during the development process to produce a combinator library over the AST
Matcher interface, however the metaprogramming resulted in monsterous compile time template
instantiation errors which were nigh on impossible to parse. During the midpoint of the project
however upstream OCLint build scripts were rebuild to accomodate Clang-3.7 which allowed the use of
C++14 standards. This allowed the development of a C++ auto closure interface over the matcher
constructs, allowing for development and utilisation of the combinator library. Whilst initially
the time constraint of developing matchers for graph traversals proved prohibitive, by using the
combinator library it appears that future development should not take as much time.

\subsubsection{CAPA LIMITATION 3}

\subsubsection{CAPA LIMITATION 4}

\pagebreak

\subsection{Extensions}
CAPA and the benchmarker both have the possibility to be extended beyond this project, potentially
by another final year student. Potential extensions are summarised here.

\subsubsection{CAPA To Capture More Patterns}
As described in the limitations, CAPA currently on captures a very small number of potentially
parallel patterns within a codebase, it is currently possible to extend the number of rules which
the AST is matched against to provide more specifics about extra test cases. Potential examples
include capturing more linear algebra operations and capturing graph traversals. With more research
and time CAPA would be able to capture an ever increasing number of potentially parallel patterns
within sequential codebases.

\subsubsection{CAPA Could Process C++}
CAPA currently only processes C files, and whilst it can process C++ there are no matchers for any
of the iterator patterns. As a result of this only a very small subset of C++ is potentially matched
by CAPA, with majority of it passing through silently. It is however possible to extend CAPA to
process C++ files due to the reliance on Clang libraries. Expansion into C++ processing would
increase the use case and increase the potential number of users of CAPA.

\subsubsection{CAPA Could Provide FPGA Analysis}
CAPA currently exclusively looks for parallelism which is suited for GPU workflows, however the tool
is general enough in nature that it is possible to extend it to look at possible FPGA applications.
Although this is outside the scope of this project, currently I am looking at the possibility of
expanding CAPA into identifying deep pipelines for FPGA development. There is clear potential for
future students to further develop CAPA in such a way that it is capable of identifying code which
is suited for running on an FPGA.

\subsubsection{CAPA-Benchmarker Could Provide More Benchmarks}
This is a natural extension to the CAPA toolset. As more patterns are identified, more benchmarks
should be created in order to continue to provide more detailed reports.


