%----------------------------------------------------------------------------------------
%	PROJECT DESCRIPTION
%----------------------------------------------------------------------------------------

\section{Project Description}
My final year project aims to create a set of static code/compiler analytics tools to help determine which algorithms within a codebase may be easily parallelized. The specific intention it to provide users with a report describing which parts of their codebase may see a potential speed-up from redeveloping them as CUDA (GPU) kernels. In order to achieve this both benchmarking and theoretical analysis is required, as well as static analysis of the C description of the algorithm itself.

\subsection{Compiler Analytics}
Given a C description of an algorithm, a report is to be generated, highlighting areas within the code that may see a potential speed-up based on matching to known GPU performant patterns. In order to achieve this, a static analysis methodology was invoked. Direct source code analysis is incredibly difficult, and suffers from a number of drawbacks, including the difficulty to parse C and (notoriously difficult) C++. Considering the time constraints involved it was decided that the analysis tool would hook into the Clang tooling library, which provides access to the Clang Abstract Syntax Tree, giving a semi-syntax invariant canvas from which to identify relevant parallel patterns.

Given the difficulty of setting up a generic build environment for the Clang tooling, I decided to fork a project named OCLint, a C, C++ and Obj-C static analysis tool that I had worked with earlier. This tool provides a framework around the Clang tooling, however most importantly it provides sophisticated build scripts which work on a variety of operating systems and distributions.

\subsubsection{OCLint Modification}
Forking the OCLint project, which is licensed under a modified BSD license has saved significant time and effort from being wasted developing a generic build system around the Clang tooling. The OCLint software provides a direct method for interacting with the Clang AST by revealing the entire Clang library from within its rule system. This increased flexibility has in turn allowed for more time to be spent developing methods to identify parallel patterns within the generated AST. All changes to the OCLint software are unrelated to its original intention and design, and as such no pull requests are likely to be lodged, and no modifications I have written, or will write are likely to move upstream. As such I will be substantially re-engineering the OCLint software into a new tool named the C Algorithm Parallelisation Analyser (CAPA).

\subsection{GPU Benchmarking}
In order to best provide theoretical performance improvements of algorithms within a codebase, an analysis of the current hardware available is significantly important. As such rather than just provide purely theoretical numbers, part of this project involves developing a simple set of GPU benchmarks which seek to show performance metrics for the identified patterns within the code analyser. This in effect means that reports generated by the analytics tool may contain specific information pertaining to the hardware availble on the current build and test system. In order to achieve the best outcome, CUDA was decided on as the framework for development.

\subsubsection{Benchmarks}
GPU's are exceptionally good at high throughput calculations, one particular example is SIMD, meaning \emph{Same Instruction Multiple Data}. The performance of GPU's and the algorithms they are particularly useful for is well under continual research, however general problem classes that GPU's are able to solve efficiently are well understood. These problem sets include algorithms that can be described by any of the following:

\begin{itemize}
\item Map
\item Fold/Reduce
\item Scan/Prefix Sum
\item Matrix Operations
\item Depth first Graph Traversal
\end{itemize}

The actual speed improvements derived from redeveloping serial code to take advantage of the massively parallel compute power of a GPU differs between each of these operations, however many serial algorithms have equivalent or more performant alternate parallel implementations. As such this project involves developing a small set of benchmarks for GPU's that determine their performance in each of these categories.

\subsubsection{Deliberately Difficult Benchmarks}
Whilst GPU's can be incredibly powerful, deliberate design decisions in the architecture of the GPU allow for significantly worse performance than one may expect from algorithms that should seemingly perform well on a GPU. There are a number of situations that may arise which decrease GPU computational efficiency including:

\begin{itemize}
\item Dispersed Global Memory Reads
\item Inefficient use of Shared Memory
\item Divergent threads (large control overhead)
\item Data dependencies
\end{itemize}

Some serial algorithms may look easily parallelizable, however they may contain one, or many of these potential inefficenciesm, and then as such they will not perform as well as one might expect on a GPU. It is therefore important in analysing potential speedups to see the performance of algorithms which \emph{appear} to be readily parallelizable, yet in practice do not perform to expectation. As such part of the project is to develop a set of benchmarks which appear to be efficient on a GPU, yet actually perform poorly.

\subsubsection{CUDA}
CUDA is Nvidia's proprietry library and toolchain for developing parallel software. There are 2 main frameworks in the GPU programming space, CUDA and OpenCL. Whilst OpenCL is a FOSS platform, the development tools are severly lacking in comparison to the CUDA toolkit, and as such it was an easy decision to follow through with the CUDA. This however has limited the performance metrics to only comparisons involving CUDA enabled graphics cards.



