%!TEX root = main.tex
%----------------------------------------------------------------------------------------
%	RE-EVALUATION OF INITIAL GOALS
%----------------------------------------------------------------------------------------

\section{Re-evaluation of Initial Goals}
In order to consolidate the current position of this project, and to best identify a pathway to completion, it's important to take another look at the initial requirements set forth in the requirements analysis. Within the requirements analysis there were a set of goals that defined the project and from which all development so far has stemmed; by looking at these requirements and evaluating the current trajectory of the project a detailed description of what is required and how it will be achieved can be compiled.

The requirements analysis broke the project down into 3 major components:

\begin{itemize}
\item GPU Benchmark Development
\item Algorithm Analytics Development
\item Optimisation Analytics Development
\end{itemize}

which then allows the breakdown of what so far has happened in the project.

\subsection{GPU Benchmark Development}
As described earlier, the importance of developing working GPU benchmarking code for known problem classes allows for better analytics and reporting in the serial algorithm analysis portions. This therefore is a key aspect of satisfactorily completing the project. The GPU benchmark development has a number of requirements that describe what the project necessitates.

\subsubsection{Requirements}

\paragraph{[FR.003]} \label{[FR.003]}
\textbf{The program shall run developed benchmark algorithms to further analytical information.}

This requirement relates directly to the overall aim of the project, which is described in the requirements just proceeding this. As the project currently stands there is only one working benchmark developed, it is a best case memory bandwidth test which requests on device memory, fills it with junk host memory, and requests theen now junk device memory be copied back to the host. This test aims to determine the peak memory bandwidth for the device, which can then be used to determine absolute optimal performance of any subsequent alogrithm.

In order to complete the project more benchmarks need to be written, to specifically cover algorithm optimisation cases identified in the serial analytics portion of the project. Necessary benchmarks are as follows: 

\begin{itemize}
\item Peak Memory Bandwidth
\item Peak Map
\item Peak Fold/Reduce
\item Peak Scan
\item Peak Matrix Multiplication
\item Peak Depth first Graph Traversal
\end{itemize}

Upon completion of these benchmarks this requirement will have been completed, interfacing and using the results of these benchmarks are a seperate requirement.

\paragraph{[OA.001]} \label{[OA.001]}
\textbf{The program shall run custom benchmark algorithms to identify GPU
performance.}

This requirement describes that the benchmarking algorithms must be utilised to identify GPU performance. In order to satisfy this requirement my intention is to produce benchmarking code for GPU performance in problem sets that are both known to be performant on a GPU as well as benchmarks that may naively appear to be performant, yet further inspection demonstrates that they are not in fact performant. This is a rather large task in and of itself, and has the potential to be an entire FYP on its own, as such significant compromises must be undertaken. In this particular case only 2 non-performant algorithms will be developed, they are:

\begin{itemize}
\item Recursive Dependent Matrix Calculations
\item Highly Divergent Hashmap Traversal 
\end{itemize}

These problem sets seem to be readily parallelizable, however they in fact exacerbate the weaknesses of the general NVidia GPU architecture (and potentially other co-processor architectures I have not worked with). The results of these benchmarks contextualise the information derived from the code analytics portion, giving rise to more useful metrics defining best and worst case performance.

\paragraph{[OA.002]}
\textbf{The program shall work on all CUDA devices.}

After careful consideration this requirement has been relaxed as it is far too strict. When writing the requirements analysis I was not as familiar with the CUDA toolchain as I am at this point of my project, and as such it is now apparent that writing Compute Capability agnostic code is a very difficult feat. In order to best satisfy the other requirements of this project I shall be limiting benchmarking code to work on CUDA capable devices of Compute Capability 3.5 and above. This compute capability was chosen as the capabilities of CUDA Cards differ significantly pre and post Compute Capability 3. This revision of the initial requirement shall save significant time and effort from being wasted in localisation and highly technical activities that benefit the overall project very little.

\paragraph{[OA.003]}
\textbf{The program shall provide comparative CPU performance metrics.}

This requirement remains as it was initially written, there is no reason why the project should not continue to include a comparison between parallel GPU benchmarks and the relevant serial CPU implementation. This information will be used within the analytics framework.

\paragraph{[OA.004]}
\textbf{The program shall provide a number of different problem class
benchmark algorithms.}

This requirement was covered and expanded under sections \ref{[FR.003]} and \ref{[OA.001]}













