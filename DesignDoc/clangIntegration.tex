%!TEX root = ./main.tex
%----------------------------------------------------------------------------------------
% CLANG INTEGRATION
%----------------------------------------------------------------------------------------

\section{Clang Integration} % Major section
CAPA utilises the Clang Libtooling library in order to perform semantic static analysis of any C
codebase, via a hook into the clang frontend action.

\subsection{Clang Compilation}
Clang is the C frontend of the llvm project, and as such it compiles source code into llvm
intermediate representation, rather than directly to ASM. This allows Clang to target a common
language leveraging the llvm compiler to perform the final compilation to a native binary. A core
tenant of the Clang toolset is that Clang is more than just a Compiler, it also a library, which
allows for third party individuals to use and extend the Clang framework in a variety of ways.
\cite{clangFeatures}
The Clang frontend compilation phase is a 4 step process, where Source code is lexed to tokens,
parsed into an AST for programatic manipulation. Semantic Analysis is performed on the AST to
identify standards compilance and enforce some optimisations, before being process by the Code
Generator which exports LLVM IR. Each of these phases in the frontend action is exposed via a public
library which can be utilised by third party tools.

\subsection{Benefits of Clang}
Clang provides a number of benefits over rolling your own source code analyser. Primarily Clang is
one of the most used C language compilers in the world, it is well maintained, well documented,
standards compliant and provides first class library support at most levels of compilation. Using
existing tooling for the heavy lifting of the project allowed for far more time to be invested in
further developing the project, rather than scaffolding the basic fundamentals. Using the Clang
libraries also allows for expansion in the future to parsing more than just C files, there is
potential to parse any file compilable by Clang, from C++, Obj-C and from Clang4.0 onwards Cuda-C
\cite{clangFeatures}.

\subsection{Integrating CAPA}\label{integrating_capa}
The Clang compiler exposes a public library for interfacing with their intermediate
compilation stage representations of the original source. For the purpose of this project it was
decided to use the exposed AST interface in order to perform the static analysis. Clang provides a
number of methods for working with the AST, namely the Visitor and Matcher interfaces. The Visitor
library utilises the visitor pattern, and a callback is undertaken upon visiting any node which
meets the requirements set forth in the visitor module. This is a useful tool, however it is not as
powerful as the Matcher interface, which allows complex grammars to be generated for highly
specific, tailored matches. CAPA utilises the ASTMatcher callback interface in order to provide complex
generic and extensible traversals of the AST. 

\includegraphicscaption{./Pictures/ClangHook.png}{CAPA Hook-In}

As CAPA is a fork of OCLint \cite{oclint} the scaffolding around the Clang integration was already
provided. CAPA accepts compile flags via the command line which are passed through to the existing
Clang compilation libraries which engage in the frontend action. CAPA hooks into the compilation
phase and provides requests after the parser has constructed the AST. At this point the matcher
interface begins to operate on the internal AST, wherein CAPA begins to analyse the provided source
files.

%------------------------------------------------


